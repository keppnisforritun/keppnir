\problemname{Eindahraðall}

Verið er að setja upp eindahraðal á Íslandi.
Hraðallinn er $N$ Planck lengdir að lengd og liggur í hring.
Nú er verið að skipuleggja tilraun þar sem skella á tveimur eindum saman.
Í byrjun er rafeind sett í hringinn og hraða seglarnir henni áfram.
Rafeindin ferðast ávallt heiltölufjölda Planck lengda áfram.
Þar sem hún er að fara hraðar og hraðar ferðast hún ávallt lengra en síðast.
Í fyrsta skrefi eru helmingslíkur að hún fari eina Planck lengd, fjórðungslíkur að hún fari tvær og svo framvegis.
Eins ef hún ferðaðist $k$ Planck lengdir síðast eru helmingslíkur að hún fari $k + 1$ næst,
fjórðungslíkur að hún fari $k + 2$ næst og svo framvegis.
Þegar eindin færist um $\geq N$ Planck lengdir í einu halda seglarnir ekki í við að halda henni inn á brautinni.
Því viljum við skella henni á aðra eind á þeirri agnarstundu sem það gerist.
Til þess þarf eindin að vera komin þar sem hún byrjaði þegar þetta gerist.
Hverjar eru líkurnar á því?

Til dæmis, ef $N = 7$ gæti hún farið tvær Planck lengdir áfram, svo þrjár og loks sex áður en hún þarf að vera komin á upphafspunkt.
Í þessu dæmi endar hún ekki þar aftur, en margir aðrir ferlar koma til greina.
Til að átta okkur á hversu oft fráhrindandi áhrifin koma til leiks höfum við áhuga á líkunum á að eindin verði á upphafspunkti þegar rannsókn lýkur.
Ef eindin er í upphafspunkti er jafn líklegt að hún fari til vinstri og að hún fari til hægri.
Ef eindin færðist um $k$ skref síðast er jafn líklegt að hún taki $k + 1$ skref og að hún taki fleiri.
Eins er jafn líklegt að hún taki $k + 2$ og hún taki fleiri, svo það eru helmingslíkur á $k + 1$, fjórðungslíkur á $k + 2$ og svo framvegis.

\section*{Inntak}
Fyrsta línan inniheldur eina tölu $1 \leq t \leq 100$, fjölda prófunartilfella.
Svo koma $t$ línur, hver með einni heiltölu $1 \leq N \leq 10^{18}$.

\section*{Úttak}
Fyrir hvert $N$ í inntaki skal prenta líkunum sem lýst er að ofan fyrir það $N$ á sinni eigin línu.
Líkurnar má rita sem fullstytt brot $p/q$.
Þar sem þessar tölur gætu verið mjög stórar skal í staðinn prenta $pq^{-1}$ mátað við $1\,000\,000\,007$,
þar sem $q^{-1}$ er margföldunarandhverfa $q$ með tilliti til mátunar við $1\,000\,000\,007$.

Til dæmis, ef $p=2$ og $q=4$, þá finnum við heiltöluna $q^{-1}$ sem uppfyllir $q \cdot q^{-1} \equiv 1 \pmod{1\,000\,000\,007}$.
Þar sem $4 \cdot 250\,000\,002 = 1\,000\,000\,008$ og $1\,000\,000\,008 \equiv 1 \pmod{1\,000\,000\,007}$ getum við reiknað út
$pq^{-1} \equiv 2 \cdot 250\,000\,002 \equiv 500\,000\,004 \pmod{1\,000\,000\,007}$.
