\problemname{Deildajöfnuður}

Rektor háskólans er að skipuleggja fjármál háskólans.
Fjármagnið kemur frá Menntamálaráðuneytinu og fylgir því vel skilgreindum ríkisstaðli.
Ráðuneytið styrkir háskólann með því að bjóða upp einhvern fjölda styrkja.
Í háskólanum eru $26$ deildir, hver þeirra merkt með enskum bókstaf.
Hver styrkur er tileinkaður sérstakri deild, og er það ráðuneytið sem ákveður deildina sem styrkurinn er tileinkaður.
Allir styrkirnir eru einnig jafn stórir.

Rektorinn getur samþykkt eða hafnað hverjum styrk fyrir sig, en strangar reglur gilda um þetta ferli.
Til að takmarka fjárhagslega ringulreið fyrir ráðuneytið, þarf rektorinn að velja eina samfellda runu af styrkjum sem hann samþykkir.
Röð styrkjana er ákvörðuð fyrirfram og er rektornum óheimilt að endurraða þeim.
Það þýðir að það getur ekki verið styrkur sem hann neitar milli fyrsta samþykkta styrksins og síðasta samþykkta styrksins.
Annars væri of erfitt fyrir ráðuneytið að skipuleggja fjárhagsáætlun sína.

Rektorinn vill hámarka samtals fjármagnið sem háskólinn fær.
Hann vill helst samþykkja alla styrkina.
Því miður er lífið hans ekki svo einfalt, þar sem krafa er að allar deildir sem fá fjármagn fái jafn mikið fjármagn.
Það þýðir að sumar deildir mega vera ófjármagnaðar.
Sumar deildir gætu jafnvel ekki verið tilnefndar fyrir neinn styrk frá ráðuneytinu.

Getur þú fundið hæsta fjölda styrkja sem rektorinn getur samþykkt sem uppfyllir þessi skilyrði?

\section*{Inntak}
Fyrsta lína inntaksins inniheldur eina heiltölu $n$, fjölda styrkja, þar sem $1 \leq n \leq 200$.
Önnur lína inntaksins inniheldur $n$ enska lágstafi, einn fyrir hvern styrk, þar sem hver stafur táknar deildina sem styrkurinn er tileinkaður.

\section*{Úttak}
Skrifaðu út eina heiltölu, hæsta fjölda styrkja sem rektorinn getur samþykkt og samt tryggt að allar fjármagnaðar deildir fái jafn mikið fjármagn.

\section*{Útskýring á sýnidæmum}
Í fyrsta sýnidæminu getur rektorinn samþykkt alla styrkina, því hver einasta deild sem kemur fyrir í inntakinu kemur fyrir nákvæmlega einu sinni.

Í öðru sýnidæminu getur rektorinn samþykkt styrkina \texttt{narnar}, þar sem sérhver deild kemur fyrir tvisvar. Athugið að þó að deildin \texttt{h}, til dæmis, komi fyrir í inntakinu, þá er hún ófjármögnuð í þessu tilviki, sem er í lagi.

Í þriðja sýnidæminu getur rektorinn samþykkt styrkina \texttt{arnanrnarnraranrna}, þar sem sérhver deild kemur fyrir $6$ sinnum.
