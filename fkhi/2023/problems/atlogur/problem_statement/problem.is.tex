\problemname{Atlögur}

Guðmundur er doktorsnemi í sögu og stundar námið í íslenskum háskóla.
Hann hefur brennandi áhuga á einvígum og háttum riddara fortíðarinnar.
Á miðöldum, og fyrr á tíð, voru atlögur milli riddara algengar.
Rannsóknir hans eru um þessa riddara og hefur hann mestan áhuga á að skrifa um sigursælasta riddarann.

Guðmundur hefur fundið handrit sem lýsir $n$ riddurum.
Í handritinu hafði riddurunum verið raðað eftir tilhneigingu þeirra að leggja til atlögu.
Fyrsti riddarinn mun því gera atlögu að öðrum riddaranum.
Sigurvegarinn mun svo gera atlögu að þriðja riddaranum, og svo framvegis.
Það er einungis ein atlaga í gangi á hverri stundu og eru alltaf einungis tveir riddarar í hverri atlögu.

Riddarinn sem lagði til atlögunnar fékk að gera fyrstu árás.
Þegar riddari $i$ skaðaði annan riddara $j$, þá lækkaði hann hraustleika hins riddarans, $h_j$, um styrk sinn $s_i$.
Ef hraustleiki hins riddarans $h_j$ var ekki lengur jákvæð tala, þá hafði riddari $j$ tapað og riddari $i$ stóð sigursæll.
Annars hélt atlagan áfram og hinn riddarinn fékk að gera árás.
Atlagan hélt áfram þar til annar riddarinn var sigursæll, og hélt hann áfram í næstu atlögu.
Allur skaði sem var skeður við riddarana var endanlegur, heilsa þeirra batnaði ekki milli atlagna.

Það gat því aðeins verið einn sem stóð sigursæll að lokum eftir allar atlögurnar.
Hvaða riddari var það sem sigraði?

\section*{Inntak}
Fyrsta lína inntaksins inniheldur eina heiltölu $n$, fjölda riddara, þar sem $1 \leq n \leq 1\,000$.
Næst fylgja $n$ línur, þar sem $i$-ta línan samanstendur af tveimur heiltölum $h_i$, hraustleika riddara $i$, og $s_i$, styrk riddara $i$, þar sem $1 \leq h_i, s_i, \leq 10\,000$.

\section*{Úttak}
Skrifaðu út eina heiltölu, vísi riddarans sem stóð sigursæll eftir allar atlögurnar.
