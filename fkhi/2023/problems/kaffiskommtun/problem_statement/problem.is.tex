\problemname{Kaffiskömmtun}

Ein mikilvægasta auðlind háskóla er kaffi.
Mjög mikilvægt er að þessi auðlind sé nýtt til fulls.

Til þess þarf að hella upp á kaffi reglulega.
Það þarf að tímasetja þessa uppáhellingu rétt því tölvunarfræðiprófessorar hafa engan afgangs tíma.
Ef ekkert kaffi er tilbúið þegar þeir fara á kaffistofuna munu þeir einfaldlega þurfa að fara aftur að kenna án þess að fá kaffi.
Þeir eru hins vegar allir mjög samviskusamir og geta hellt upp á nýja könnu af kaffi þegar þeir mæta á kaffistofuna, jafnvel eftir að hafa fyllt sinn eiginn bolla.
Eftir að byrjað er að hella upp á nýja könnu af kaffi þurfa að líða $T$ sekúndur áður en nýja kannan af kaffi er tilbúin.
Ný kanna af kaffi getur fyllt $C$ kaffibolla.

Hver er hámarksfjöldi manns sem geta fengið sér kaffi ef við skipuleggjum uppáhellingar fullkomlega?
Athugið að stundum er best að hella afgangskaffi úr könnunni og hella upp á nýtt kaffi þó það sé syndsamleg sóun á mikilvægri auðlind.
Kannan er tóm í upphafi.

\section*{Inntak}
Fyrsta lína inntaksins innheldur þrjár heiltölur $N$,
fjölda prófessora sem fara í kaffipásu, 
þar sem $1 \leq N \leq 10^6$, $C$,
fjölda bolla sem ein kanna getur fyllt,
þar sem $1 \leq C \leq 500$, og $T$,
fjölda sekúndna sem það tekur fyrir nýja könnu að vera tilbúna,
þar sem $1 \leq T \leq 10^9$.
Næst kemur ein lína með $N$ heiltölum $1 \leq t_1 \leq t_2 \leq \dots \leq t_N \leq 10^9$, tímasetningarnar þegar prófessorarnir
fara í kaffipásu.

\section*{Úttak}
Prenta skal hámarksfjölda prófessora sem geta fengið kaffi þegar þeir fara í pásu.

