\problemname{Majór Skýrsla}

Þér hefur borist það verkefni að raða upp og undirbúa skýrslu fyrir breskan majór. Þessi majór stundar
stærðfræði í frítíma sínum og kann því vel að meta mynstur og tölur í öllu sem hann sér. Þegar honum
berst bunki af pappírum stundar hann oft að skoða vísi bunkans. Hver blaðsíða inniheldur einhverjar upplýsingar,
og eru þessar upplýsingar mismikilvægar. Til að reikna vísinn fer hann í gegnum bunkann
frá toppi til botns og leggur saman öll blaðsíðutöl þar sem blaðsíðan er mikilvægari en sú sem á eftir kemur.
Svo ef þriðja blaðsíðan sem hann les er mikilvægari en sú fjórða bætir hann þremur við vísinn.
Getur þú raðað öllum $n$ blaðsíðum skýrslunnar svo vísi bunkans sé $k$, uppáhalds tala majórsins?

\section*{Inntak}
Fyrsta og eina lína inntaksins innheldur tvær heiltölur $n$, fjöldi blaðsíðna, þar sem $1 \leq n \leq 10^5$,
og $k$, uppáhaldstala majórsins, þar sem $0 \leq k \leq \frac{n \cdot (n-1)}{2}$.

\section*{Úttak}
Prentið tölurnar $1, 2, \dots, n$ í einhverri röð á einni línu með einu bili milli talna. Fremsta talan samsvarar
síðunni sem majórinn les fyrst og aftasta síðunni sem hann les síðast.
Minnst mikilvæga blaðsíðan er táknuð af $1$,
næst minnst mikilvæga blaðsíðan er táknuð af $2$,
og svo framvegis upp í $n$ sem er mikilvægasta blaðsíðan.
Úttakið þitt er talið rétt ef bunkinng sem það lýsir hefur vísinn $k$.

Til dæmis ef $n=5$ og $k=4$, þá væri úttakið \texttt{5 2 4 1 3} talið rétt því fyrsta og þriðja blaðsíðan mikilvægari en sú næsta, svo vísirinn er $1+3=4$.
