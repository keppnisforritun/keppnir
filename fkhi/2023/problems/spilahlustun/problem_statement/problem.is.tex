\problemname{Spilahlustun}

Veskið hans Jörmunreks er orðið ansi létt nú undir lok náms hans, svo hann er að leita að frumlegum aðferðum til
að laga þann vanda. Eftir að hafa setið líkindafræðiáfanga veit hann að flest spil upp á peninga með spilastokk eru ekki í hans hag, en hann
er með smá bragð í huga. Hann er búinn að kaupa sér lítinn hljóðnema sem hann festir undir borðið sem hann spilar við.
Til þess að sjá til þess að stokkunin sé sanngjörn er lítil stokkunarvél notuð, vél sem er nægilega hljóðlát til að
mannfólk heyri ekkert í henni. Hún stokkar með því að taka hlut af spilunum, snúa þeim við svo þau séu í öfugri röð, og
svo endurtaka þetta nógu oft. Vélin passar svo einfaldlega upp á að snúa efsta spilinu við rétt áður en spilið er gefið ef 
nauðsyn krefur. Í gegnum hljóðnemann getur Jörmunrekur mælt hversu langt rafmótorar tækisins færast áður en spilum er snúið,
og hvað armurinn breikkast vítt til að tína þau upp. Þar með veit hann hvaða spilum er snúið við í hvert sinn. Hann þarf
hins vegar að einbeita sér að spilinu, svo hann þarf forrit sem segir honum röð spilana í lok stokkunar.
Getur þú búið til slíkt forrit?
Spilin eru númeruð frá $1$ upp í $n$.
Í upphafi er efsta spilið $1$, næstefsta spilið er $2$, og svo framvegis, alveg þar til komið er að neðsta spilinu, sem er $n$.
Öll spilin snúa niður í upphafi.

\section*{Inntak}
Fyrsta lína inntaksins inniheldur tvær heiltölur $n$, fjölda spila, þar sem $1 \leq n \leq 5 \cdot 10^5$, og $q$, fjölda snúninga í stokkuninni, þar sem $1 \leq q \leq 5 \cdot 10^5$.
og $q$ er fjöldi snúninga í stokkuninni. Svo koma $q$ línur sem hver lýsa einni viðsnúningu í formi tveggja talna $1 \leq 
i \leq j \leq n$. Þetta merkir að frá og með $i$-ta efsta spilinu til og með $j$-ta efsta spilinu sé snúið sem einni heild. Snúningarnir
eru gefnir í þeirri röð sem vélin framkvæmir þá.

\section*{Úttak}
Skrifaðu út númerið á sérhverju spili frá toppi til botns, eins og þau eru röðuð í lok stokkunar.
Ef spil $k$ snýr upp í lokin skaltu skrifa út $-k$ í staðin fyrir $k$.

