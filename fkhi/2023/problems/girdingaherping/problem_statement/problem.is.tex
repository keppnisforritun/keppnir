\problemname{Girðingaherping}

Búið er að leggja niður girðingu fyrir nýrri byggingu. Hins vegar hafa borist fregnir um að skera eigi niður, svo minnka
þarf lóðina. Stærðfræðingarnir höfðu eitthvað orð á því að herpingar væri hentugar til að gera slíkt, svo ætlunin er að
herpa jaðar svæðisins þannig að flatarmálið helmingast. Lítum á jaðar svæðisins sem runu línustrika sem byrja og enda á
sama stað án þess að tvö línustrik skerist, nema í endapunktum aðlægra línustrika. En þegar þetta var prófað kom í ljós
að það að draga allt saman um einn punkt gat ollið því að sumir hlutar enduðu utan við upprunalegu lóðina, sem gengur
ekki. Eftir langa fundi við reynda rúmfræðinga er sú lausn lögð fram að hliðra hverju línustriki um þvervigur sinn inn
á við og stytta eða lengja það svo mátulega þannig að það leggist að aðlægu línustrikunum. Þá þarf einfaldlega að finna
út úr því hvað hliðra á hverju striki langt svo að flatarmálið helmingast.

Þetta gengur vel fyrir sumar teikningar en ekki aðrar, því það sem getur gerst er að línustrik sem ekki eru aðlæg renni
svo langt að þau rekist utan í hvort annað. Þegar þetta gerist segjum við að hliðrunin sé ógild. Enn fremur er hliðrun
ógild ef hún hliðrar girðingum gegnum hvora aðra. Ef hvorugt af þessu gerist telst hliðrunin gild.

\section*{Inntak}
Fyrsta lína inntaksins inniheldur eina heiltölu $n$, fjölda hornpunkta línustrikanna sem lýsa lóðinni, sem uppfyllir
$3 \leq n \leq 5 \cdot 10^4$.
Næstu $n$ línur innihalda hornpunkta lóðarinnar, gefnir í jákvætt áttaðri röð meðfram jaðrinum. Hverjum punkti er lýst
með tveimur heiltölum $x, y$ með bili á milli sem uppfylla $-10^9 \leq x, y \leq 10^9$. $x$ lýsir $x$-hniti punktsins
og $y$ lýsir $y$-hniti punktsins. Engir þrír punktar í röð liggja á sömu línu.

\section*{Úttak}
Ef engin gild hliðrun helmingar flatarmálið, prentið ,,Omogulegt!''. Prentið annars fjarlægðina sem hliðra á hverju
línustriki. Svar telst rétt ef bein eða hlutfallsleg skekkja þess frá réttu svari er mest $10^{-6}$. Ef til er gild hliðrun
sem helmingar flatarmálið er einnig til hliðrun sem tekur flatarmálið í $0.5 - 10^{-5}$ sinnum upprunalega flatarmálið.

