\problemname{Einfasa Eindahraðall}
\illustration{0.5}{1862}{Mynd eftir Randall Munroe, \href{https://xkcd.com/1862/}{xkcd.com}}

Eins og kom fram í síðustu keppni er búið að setja upp
íslenskan eindahraðal. Frá því síðast er einhver búinn að
taka það að sér að koma honum yfir á einfasa rafmagn, því
hinir fasarnir voru að trufla mælingar samkvæmt einhverjum
rafverkfræðingi. Álpappírshatturinn var undarlegur, en tækið
virkar alla vega. Nú þarf bara að vinna úr gögnunum sem
eindahraðallinn spýtti út.

Eindahraðallinn, eins og nafnið gefur til kynna, hraðar
eindum. Þetta er gert til að skella saman eindum með miklum
krafti til að mynda og greina nýjar og sjaldgæfar eindir.
Þessum eindum er lýst út frá ýmsum skammtafræðilegum
eiginleikum, eins og sjá má á myndinni. 

Með því að endurskala hluti rétt má ráðstafa hlutum sem svo
að öllum þessum skammtafræðilegum eiginleikum megi lýsa með
heiltölum, þar sem hver eiginleiki hefur eitthvað
lágmarks- og hámarksgildi og getur tekið sérhvert gildi
þar á milli.

Eindahraðallinn spýtir út þessum skammtafræðilegu gildum
fyrir sérhverja eind sem hún mælir. Saman gefa öll þessi
gildi gerð eindarinnar, sem er þá runa heiltalna, sem 
ákvarða eindina ótvírætt. En þar eðlisfræðingarnir
eru að leita að tiltekinni eind sem strengjafræði þeirra
spáir fyrir um þarf að vinna úr gögnunum aðeins.

Fyrir hverja tilgátu sem eðlisfræðingarnir hafa vilja þeir
vita hversu margar eindir mældust sem hafa sérhvern
skammtafræðilegan eiginleika innan viss bils. Til dæmis
ef lýsa mætti eindum út frá hleðslu, massa og keim gætu
eðlisfræðingarnir beðið um fjölda einda með hleðslu
nákvæmlega $-3$, massa frá $2$ til $5$ og keim frá $-1$
til $1$. Ein gerð einda sem passar við þetta væri þá
til dæmis $(-3, 4, 0)$.

\section*{Inntak}
Fyrsta lína inntaksins inniheldur jákvæða heiltölu $n$,
fjöldi skammtafræðilegra eiginleika sem eindahraðallinn
mælir. Gefið er að $1 \leq n \leq 10$.
Næst fylgir lína með $n$ pörum heiltalna $l_i, r_i$ þar sem
$l_i$ gefur lágmarksgildi og $r_i$ hámarksgildi
$i$-ta skammtafræðilega eiginleikans. Gildin eru gefin í
röðinni $l_1, r_1, l_2, r_2$ og svo framvegis, aðskilin
með bili. Gefið er að $-10^9 \leq l_i \leq r_i \leq 10^9$ 
fyrir öll $i$ og að eindirnar geti haft mest $10^6$ ólíkar
gerðir í heildina.
Næst fylgir lína með einni jákvæðri heiltölu $p$, fjöldi
einda sem eindahraðallinn mældi. Gefið er að
$1 \leq p \leq 10^5$.
Næst fylgja $p$ línur þar sem $i$-ta línan lýsir $i$-tu
eindinni sem eindahraðallinn mældi.
Á $i$-tu línu eru $n$ gildi $x_j$ þar sem $x_j$ lýsir
$j$-ta skammtafræðilega eiginleika $i$-tu eindarinnar sem
uppfyllir $l_j \leq x_j \leq r_j$, þ.e. $i$-ta línan
gefur gerð $i$-tu eindarinnar.
Næst kemur ein lína með jákvæðri heiltölu $q$, fjöldi
fyrirspurna frá eðlisfræðingunum. Gefið er að
$1 \leq q \leq 10^5$.
Loks koma $q$ línur til viðbótar, þar sem $i$-ta þeirra
lýsir $i$-tu fyrirspurn eðlisfræðinganna.
Á $i$-tu línu eru $n$ pör heiltalna $a_j, b_j$ þar sem
$a_j$ gefur lágmarksgildi og $b_j$ hámarksgildi
$j$-ta skammtafræðilega eiginleikans. Gildin eru gefin í
röðinni $a_1, b_1, a_2, b_2$ og svo framvegis, aðskilin
með bili. Gefið er að $-10^9 \leq a_j \leq b_j \leq 10^9$
fyrir öll $j$.

\section*{Úttak}
Fyrir hverja fyrirspurn eðlisfræðinganna skal prenta eina
heiltölu á sinni eigin línu, fjölda einda í inntaki sem
hafa sérhvern skammtafræðilegan eiginleika innan marka
fyrirspurnarinnar. Prenta skal svörin í sömu röð og
fyrirspurnirnar eru gefnar.
