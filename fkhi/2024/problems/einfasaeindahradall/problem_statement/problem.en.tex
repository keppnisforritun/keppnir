\problemname{Einfasa Eindahraðall}
\illustration{0.5}{1862}{Image by Randall Munroe, \href{https://xkcd.com/1862/}{xkcd.com}}

As was covered in the last contest, an Icelandic particle
accelerator has been built. Since last time the accelerator
has been converted to use single phase power, since according
to some electrical engineer the other phases were
interfering with measurements. The tin foil hat was an unusual
fashion statement, but the machine works in any case. Now the
only thing left is to process all the data it spits out.

The particle accelerator, as the name suggests, accelerates
particles. This is done to then smash particles together with
great speed to create and analyse new and rare particles.
These particles are described using various quantum
properties, as shown on the image.

By scaling things properly all of these quantum properties
can be described by integer values, where each property has
some minimum and maximum possible value, and can take every
integer value there between.

The particle accelerator spits out these quantum values
for each particle it measures. Together these values
determine the type of the particle, which is a sequence
of integers, which uniquely determine the particle.
But since the physicists are
looking for specific particles given by string theory
conjectures, the data has to be processed somewhat.

For each conjecture the physicists have they want to know
how many particles were measured that fit their description,
meaning that each of its quantum properties fits in some
given interval. For example if particles were given by
charge, mass and flavour the physicists might want to know
the number of particles with charge exactly $-3$, mass from
$2$ to $5$ and flavour from $-1$ to $1$. An example of a
particle type satisfying these bounds is $(-3, 4, 0)$.

\section*{Input}
The first line of input contains a positive integer $n$,
the number of quantum properties the particle accelerator
measures. You may assume $1 \leq n \leq 10$.
Next there is a line with $n$ pairs of integers $l_i, r_i$
where $l_i$ is the minimum value and $r_i$ is the maximum
value of the $i$-th quantum property. The values are given
in the order $l_1, r_1, l_2, r_2$ and so on, separated by
spaces. You may assume $-10^9 \leq l_i \leq r_i \leq 10^9$
for all $i$ and that the particles can only have at most
$10^6$ different types in total.
Next there is a line with a positive integer $p$, the number
of particles the accelerator measured. You may assume
that $1 \leq p \leq 10^5$.
Next there are $p$ lines where the $i$-th lines describes
the $i$-th particle measured.
The $i$-th line contains $n$ values $x_j$ where $x_j$
gives the value of the $j$-th quantum property of the
$i$-th particle measured. It satisfies 
$l_j \leq x_j \leq r_j$. This means the $i$-th line gives
the type of the $i$-th particle measured.
Next there is a line with a positive integer $q$, the
number of conjectures from the physicists. You may assume
$1 \leq q \leq 10^5$.
Finally there are $q$ more lines where the $i$-th line
describes the $i$-th conjecture.
The $i$-th line contains $n$ pairs of integers $a_j, b_j$
where $a_j$ is the minimum value and $b_j$ the maximum value
the $j$-th quantum property can be. The values are given in
the order $a_1, b_1, a_2, b_2$ and so on, separated by spaces.
You may assume $-10^9 \leq a_j \leq b_j \leq 10^9$ for all
$j$.

\section*{Output}
For each conjecture print a single integer on its own line,
the number of particles in the input falling within the
bounds of the conjecture for every quantum property. 
Print the answers in the same order as the conjectures are
given in the input.
