\problemname{Chili COM Carne}
\illustration{0.5}{1205}{Mynd eftir Randall Munroe, \href{https://xkcd.com/1205/}{xkcd.com}}

Forritarar þekkja allir COM, sem stendur auðvitað fyrir
Cost Of Maintenance, og ekkert annað.
Oft er hægt að halda slíkum kostnaði niðri með því að láta
tölvur sjá um viss verk, en margir forritarar eru aðeins
of æstir í að ganga í slíkt og enda með því að eyða meiri
tíma í að forrita tölvuna til að sjá um verkefnið en það
hefði tekið að gera það handvirkt út alla ævi.

Því þarf nú að búa til forrit sem sjálfkrafa sér um að
ákvarða hvort það sé þess virði að skrifa forrit sem gerir
hluti sjálfkrafa. Þar sem þetta forrit er ekki til ennþá er
ekki víst hvort það sé þess virði að skrifa það, en það verður
bara að koma í ljós.

Við mælum tímann okkar í nokkrum einingum þar sem stærsta
einingin eru ár. Í einu ári eru svo $52$ vikur. Í hverri
viku eru $5$ vinnudagar, og í hverjum vinnudag eru $8$
vinnustundir. Loks eru náttúrulega $60$ mínútur í hverri
klukkustund og $60$ sekúndur í hverri mínútu.

Að því gefnu hversu oft þarf að sinna verkefninu, hvað það
tekur langan tíma að sinna því, og hversu lengi tæki að
búa til forrit sem sér um það sjálfkrafa þarf þá að segja
hversu mikinn tíma það gæti sparað eftir fimm ár. 
Sparnaðurinn er mældur frá og með að forritið er tilbúið.

\section*{Inntak}
Fyrsta línan gefur hversu oft þarf að sinna verkefninu
á forminu \texttt{n sinnum daglega}. Ef $n$ er $1$ stendur
\texttt{sinni} í staðinn fyrir \texttt{sinnum}. Í staðinn
fyrir \texttt{daglega} getur einnig staðið \texttt{vikulega}
eða \texttt{arlega} (árlega).
Næsta lína gefur hversu langan tíma það tekur að sinna
verkefninu á forminu \texttt{n sekundur}. Í staðinn fyrir
\texttt{sekundur} getur einnig staðið \texttt{minutur},
\texttt{klukkustundir}, \texttt{dagar}, \texttt{vikur},
\texttt{ar} (ár). Ef $n = 1$ kemur eintöluform orðsins í
staðinn, s.s. eitt af \texttt{sekunda}, \texttt{minuta},
\texttt{klukkustund}, \texttt{dagur}, \texttt{vika} eða
\texttt{ar} (óbreytt).
Loks fylgir þriðja og síðasta línan sem gefur hversu langan
tíma tekur að búa til forrit sem sér um verkefnið sjálfkrafa.
Þetta er gefið á sama formi og línan á undan.
$n$ er jákvæð heiltala jöfn í mesta lagi $10$ í
öllum tilfellum að ofan.
Athugum að þar sem fleiri en ein manneskja getur sinnt
verkefninu er mögulegt að það taki samtals meir en fimm ár 
að sinna verkefninu næstu fimm árin.

\section*{Úttak}
Prenta skal fjölda sekúndna sem forritið myndi spara yfir $5$
ár. Ef það að skrifa forritið tekur lengri tíma en að gera
verkefnið handvirkt næstu $5$ ár skal í staðinn prenta
\texttt{Borgar sig ekki!}.
