\problemname{Chili COM Carne}
\illustration{0.5}{1205}{Image by Randall Munroe, \href{https://xkcd.com/1205/}{xkcd.com}}

Programmers all know COM, which is of course an abbreviation
for Cost Of Maintenance and has no other interpretations.
Often such costs can be lowered by automating certain tasks,
but many programmers are overzealous in this regard and
end up spending more time programming an automatic solution
than it would have taken them to do the task by hand for
the rest of their lives.

To rectify this a program has to be made that can 
automatically determine whether it is worth it to write
a program to automate things. As this program does not yet
exist we can't say whether it's worth it to write this
program, so that'll just have to be revealed in due time.

We measure time using a few different units, with the largest
one being a year. A single year has $52$ weeks. Each week
has $5$ work days and each work day has $8$ work hours.
Finally there are of course $60$ minutes to an hour and
$60$ seconds to a minute.

Given how often the task has to be done, how long it takes
to do it, and how long it would take to automate it, make a
program that can tell how much time it would save to
automate the task over the course of the next five years.
The saved time is measured starting from the program being
finished.

\section*{Input}
The first line gives how often the task needs to be performed,
given as \texttt{n sinnum daglega} where $n$ is an integer
and \texttt{daglega} means daily. If $n$ is one the second
word is replaced with \texttt{sinni}. The third word can
also be \texttt{vikulega} which means weekly or
\texttt{arlega} which means yearly.
The next line gives how long it takes to perform the
task each time, given as \texttt{n sekundur} where $n$
is an integer and \texttt{sekundur} means seconds.
Instead of \texttt{sekundur} the second word can also
be \texttt{minutur} (minutes), \texttt{klukkustundir}
(hours), \texttt{dagar} (days), \texttt{vikur} (weeks),
\texttt{ar} (years).
If $n = 1$ the input contains the singular form of the word
instead, i.e. one of \texttt{sekunda}, \texttt{minuta},
\texttt{klukkustund}, \texttt{dagur}, \texttt{vika} or
\texttt{ar}, given in the same order as above.
Finally the third and last line gives how much time it
would take to automate the task, given in the same format
as the line before.
$n$ is always a positive integer with value at most $10$.
Note that since the task might be taken care of by more than
a single person it is possible that it takes more than five
years to perform the task over the next five years in total.

\section*{Output}
Print the number of seconds the program would save over $5$
years. If automating the task takes strictly more time than
doing the task manually over the next $5$ years, instead print
\texttt{Borgar sig ekki!}.
