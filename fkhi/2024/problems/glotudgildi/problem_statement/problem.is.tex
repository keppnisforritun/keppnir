\problemname{Glötuð Gildi}
\illustration{0.5}{544}{Mynd eftir Randall Munroe, \href{https://xkcd.com/544/}{xkcd.com}}

Háskólar Íslands hafa ákveðið að halda stórt Nim-mót.
Það hafa $n$ lið skráð sig, og mun hvert lið keppa við
sérhvert annað lið nákvæmlega einu sinni.
Eins og er vel þekkt eru engin jafntefli í Nim, svo
í hverjum leik sigrar annað liðið og það fær $1$ stig.
Í lokin er búið að safna saman stigum liðanna, en það
gleymdist að halda utan um hver sigraði hvern.
Þetta er ekki gott, svo Jörmunrekur ætlar að reyna giska
á hverjar niðurstöðurnar voru.
Líkurnar á að hann hafi rétt fyrir sér eru þá háðar á
hversu marga vegu þessi stigatafla hefði getað myndast.
Því þarf að komast að því sem fyrst!

Tökum sem dæmi mót með lið $A, B, C$ og stigatöflu $1, 1, 1$.
Það gæti verið að $A$ vann $B$, $B$ vann $C$ og $C$ vann $A$.
En það kemur einnig til greina að $A$ vann $C$, $C$ vann $B$
og $B$ vann $A$.
Sjá má að þetta eru möguleikarnir, svo í þessu tilfelli
væri svarið $2$.

\section*{Inntak}

Inntakið byrjar á einni heiltölu $0 \leq n \leq 16$.
Svo fylgja $n$ heiltölur á næstu línu, sérhver þeirra
er einnig minnst $0$ og mest $16$.
Heiltala númer $i$ táknar hversu marga leiki
$i$-ta liðið vann.

\section*{Úttak}

Prentið á hversu marga vegu mótið gæti hafa farið.
Tvær leiðir teljast ólíkar ef eitthvert lið vann annað
í annarri leiðinni, en ekki í hinni.
