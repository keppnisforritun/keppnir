\problemname{Austan Atlantshafs}
\illustration{0.5}{2901}{Mynd eftir Randall Munroe, \href{https://xkcd.com/2901/}{xkcd.com}}

Margar borgir og aðrar stofnanir eiga það til að reisa
styttur af ýmsum gerðum til að vekja athygli.
Í þessu samhengi eru stærri styttur auðvitað betri, því
stærri stytta því meiri athygli.
Enn betra er ef styttan er sú stærsta á stóru svæði í
kringum punktinn þar sem hún var reist.
Sérhver stofnun vil því auglýsa styttuna sína sem ,,stærsta
styttan í X`` þar sem X er stærsta nefnda svæðið sem gerir
þetta að sannri staðhæfingu.
Til dæmis ef stytta væri stærst í allri Evrópu væri það mun
betri auglýsing en að segja að það sé stærsta stytta Íslands,
þó það væri auðvitað einnig satt ef hún væri staðsett þar.
Athugið að ef til dæmis Þýskaland og Frakkland hefðu bæði
jafn stórar stærstu styttur, þá væri hvorug þeirra stærsta
stytta Evrópu.

\section*{Inntak}
Fyrsta lína inntaksins inniheldur eina heiltölu $n$, fjöldi
landsvæða með $1 \leq n \leq 100,000$.
Næst koma $n$ línur þar sem hver þeirra lýsir einu landsvæði.
$i$-ta línan inniheldur streng $s_i$, streng $t_i$ og jákvæða
heiltölu $x_i$, aðskilin með bilum.
$s_i$ gefur nafn $i$-ta svæðisins og $t_i$ gefur nafn 
svæðisins sem $i$-ta svæðið er innihaldið í.
$t_i$ er alltaf eitt af landsvæðunum sem kemur fyrir í
inntakinu.
$x_i$ gefur svo loks hæð styttunar $i$-ta svæðinu, þar sem
$1 \leq x_i \leq 10^9$ ef það er stytta þar og $x_i = -1$ annars.
Ef svæðið inniheldur önnur svæði mun $x_i = -1$, en $x_i > 0$
annars.
Engin tvö ólík svæði hafa sama nafn.

Sérhver strengur í inntaki er af lengd mest $20$ og inniheldur
bara enska lágstafi.
Samtals lengd allra strengja í inntaki verður mest $10^6$.
Fyrsta svæðið verður ávallt \texttt{jord} og inntakið segir
að það sé innihaldið í sjálfu sér. 
Ekkert annað svæði er innihaldið
í sjálfu sér, hvorki beint né gegnum milliliði.

\section*{Úttak}
Fyrir hvert landsvæði með styttu, prentið nafn stærsta 
svæðisins þar sem stytta þess er ennþá stærst. Svæði er 
talið stærra en annað svæði ef það inniheldur hitt svæðið.
Prenta skal hvert nafn á sinni eigin línu, og gefa svörin
í sömu röð og svæðin eru gefin í inntaki.
