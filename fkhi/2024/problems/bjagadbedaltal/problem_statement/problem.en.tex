\problemname{Bjagað Beðaltal}
\illustration{0.5}{2435}{Image by Randall Munroe, \href{https://xkcd.com/2435/}{xkcd.com}}

At the end of every programming contest, statistics and fun
facts are collected to show the contestants afterwards.
Often average number of solves, or average number of tries
to solve a given problem are calculated.
Allegedly, someone at the statistics department said
that arithmetic averages are very uninteresting.
To make the contest data analysis exciting enough for
the statisticians the contest hosts have decided to use
a new exciting method to calculate averages.
Instead of following the good example set by Hölder in
how to generalise averages, the example of Randall Munroe
is followed instead.
That is where the geothmetic meandian comes in.
As the image shows, the geothmetic meandian is calculated
in the following way:

Let $n$ be an odd number.
A sequence of numbers $x_1, x_2, \dots, x_n$ is sorted resulting in the sequence $y_1, y_2, \dots, y_n$.
The original sequence is then replaced by the
sequence 
\[
    n^{-1}\sum_{i = 1}^n y_i, \sqrt[n]{\prod_{i=1}^n y_i}, y_{(n+1)/2}
\]
This is then repeated on the new sequence resulting in another sequence.
If the repetition is performed infinitely often the sequence converges to a single repeated value, this
repeated value is then the geothmetic meandian.

\section*{Input}

The input begins with a single odd integer 
$1 \leq n \leq 10^5$. 
The next line contains $n$ real values $x_1, \dots, x_n$.
For all $i$ it is guaranteed that $0 < x_i \leq 10^9$.
All real values have at most $6$ digits after the decimal.

\section*{Output}

Print the geothmetic meandian of $x_1, \dots, x_n$.
The answer is considered correct if its absolute or relative
error from the true answer is at most $10^{-5}$.
