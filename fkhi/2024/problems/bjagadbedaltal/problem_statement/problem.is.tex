\problemname{Bjagað Beðaltal}
\illustration{0.5}{2435}{Mynd eftir Randall Munroe, \href{https://xkcd.com/2435/}{xkcd.com}}

Eftir lok hverrar forritunarkeppni þarf að taka saman gögn
til að sýna keppendum skemmtilegar staðreyndir um hvernig
keppendum gekk að leysa dæmin. Oft er verið að skoða hversu
margir leystu dæmi að meðaltali, eða hversu margar tilraunir
þurfti til þess að meðaltali.
En venjulegt meðaltal er svo óspennandi, heyrðist einhver
segja á tölfræðistofunni. 
Til þess að keppnin sé nægilega spennandi fyrir tölfræðingana
er ákveðið að nota nýja spennandi leið til að reikna
meðaltal $n$ talna.
Frekar en að fylgja fordæmi Hölders og alhæfa meðaltöl á
gáfulegan hátt, er ákveðið að fylgja fordæmi Randall Munroe.
Þar kemur rúmjulega miðtalið til sögunnar.
Eins og myndin gefur til kynna er rúmjulega miðtalið
reiknað með eftirfarandi hætti:

Látum $n$ vera oddatölu.
Runu af tölum $x_1, x_2, \dots, x_n$ er raðað og þá fæst runan $y_1, y_2, \dots, y_n$.
Upprunalegu rununni er svo skipt út fyrir
\[
    n^{-1}\sum_{i = 1}^n y_i, \sqrt[n]{\prod_{i = 1}^n y_i}, y_{(n + 1)/2}
\]
Þetta er svo endurtekið fyrir nýju rununa sem leiðir til annarrar runu.
Ef endurtekningin er framkvæmd óendanlega oft stefnir runan á endurtekningu af sama gildinu, og
það endurtekna gildi er rúmjulega miðtalið.

\section*{Inntak}

Inntakið byrjar á einni odda heiltölu $1 \leq n \leq 10^5$.
Á næstu línu fylgja $n$ rauntölur $x_1, \dots, x_n$.
Fyrir öll $i$ gildir $0 < x_i \leq 10^9$.
Öll gildin verða með mest $6$ stafi eftir kommu.

\section*{Úttak}

Prentið rúmjulega miðtal $x_1, \dots, x_n$.
Svar telst rétt ef hlutfallsleg eða bein skekkja þess
frá réttu svari er mest $10^{-5}$.
