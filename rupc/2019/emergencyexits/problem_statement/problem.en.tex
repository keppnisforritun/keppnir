\problemname{Emergency Exits}

\illustration{0.4}{exit}{An emergency exit.}%
The university is set to undergo a comprehensive quality inspection next month.
The set of requirements that will be checked are known in advance, and the
university has been going through the list, making sure everything is in order.

Under the ``Fire Safety'' section there is a requirement concerning emergency
exits that they are having a hard time assessing. It states that it must be
possible to reach an emergency exit from any location within the university
building in a reasonable time.

You have been asked to help with assessing the current state of these emergency
exits. The university has provided you with a graph representation of the
building, where each location in the building is represented as a vertex, and
each pathway is represented as a weighted directed edge from one
location to another. The weight of an edge represents the time in seconds
required to travel along that pathway. Note that each pathway can only be
traveled in one direction.

Given at which locations an emergency exit is present, determine the maximum
time required to reach the closest emergency exit from any location in the
building.

\section*{Input}
The input consists of:
\begin{itemize}
    \item One line with three integers $n$, $m$ and $k$ ($1 \le k \le n \le 2\cdot
    10^5$, $0 \le m \le 2\cdot 10^5$), the number of locations, pathways and
    emergency exits.
    \item One line with $k$ integers, the distinct locations of the emergency
    exits.
    \item $m$ lines, the $i$th of which contains three integers $u_i$, $v_i$ and
    $s_i$ ($1 \le u_i,v_i \le n$, $0 \le s_i \le 10^6$, $u_i \neq v_i$),
    representing a unidirectional pathway from location $u_i$ to location $v_i$
    that takes $s_i$ seconds to travel along.
\end{itemize}

No two pathways have the same source and destination.

\section*{Output}
Output the minimum time, in seconds, required to reach the closest emergency exit
from any location in the building. If it is not possible to reach an emergency
exit from all locations in the building, output ``\texttt{danger}''.

