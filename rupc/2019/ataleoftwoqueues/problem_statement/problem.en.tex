\problemname{A Tale of Two Queues}

\illustration{0.4}{queues-small}{The two queues.}%
Following yet another all-nighter of studying patterns in permutations at the
university, Unnar is both exhausted and starving. Fortunately it is almost
noon, and the cafeteria has started serving lunch.

After heading downstairs, Unnar is sad to see that the cafeteria is already
full of hungry students, and that the two queues towards the two registers are
already quite long. Although he would prefer shorter queues, years of eating at
the cafeteria has made Unnar an expert at estimating how long different
individuals take to pay for their food at the register.

The methods Unnar uses to perform these very accurate estimations require years
of training to even begin to understand, but they are based on observations
such as whether the individual has their credit card or cash ready, the amount and
cost of the items they intend to purchase, and whether they are staff members.

After making his complex estimations for each individual in each queue, Unnar
would like to know which queue he should enter in order to get to the register
as quickly as possible, assuming his estimations are correct (which they always
are!). At this point his sleep deprivation is really starting to kick in, and
he asks you to help him with this final task.

\section*{Input}
The input consists of:
\begin{itemize}
    \item One line with two integers $n$ and $m$ ($1 \le n,m \le 5\,000$), the number
    of individuals in the left and right queues.
    \item One line with $n$ integers, the $i$th of which represents the
    estimated time, in seconds, for the $i$th individual in the left queue.
    \item One line with $m$ integers, the $i$th of which represents the
    estimated time, in seconds, for the $i$th individual in the right queue.
\end{itemize}

Individuals are listed in their queue order, with the next in queue being
listed first.

\section*{Output}
If it is quicker for Unnar to enter the left queue, output ``\texttt{left}''.
If it is quicker for Unnar to enter the right queue, output ``\texttt{right}''.
If it does not matter which queue Unnar enters, output ``\texttt{either}''.

